%% LyX 2.3.3 created this file.  For more info, see http://www.lyx.org/.
%% Do not edit unless you really know what you are doing.
\documentclass[11pt]{article}
\usepackage[latin9]{inputenc}
\usepackage{url}
\usepackage{amsmath}

\makeatletter
%%%%%%%%%%%%%%%%%%%%%%%%%%%%%% User specified LaTeX commands.
%Gummi|065|=)
\usepackage{url}
\title{\textbf{Capstone Proposal}}
\author{Allyson Julian}
\date{December 30, 2019}

\makeatother

\begin{document}
\maketitle

\section{Introduction}

Advances in agricultural technology over the past 30 years have made
it easier for farmers to manage their farms, particularly when the
farms are comprised of multiple fields greater than 1000 acres in
size. The use of GPS on aerial imagery, drones, etc. have been instrumental
in precision agriculture\cite{liakos_machine_2018}.

But one of the challenges of modern farming is the detection of plant
diseases. For large farms in particular, it is time-consuming for
a farmer to manually check each growing plant for disease. It can
potentially be more cost-effective to diagnose plant diseases with
automated tools \cite{fujita_basic_2016}.

\section{Problem Statement}

Farmers need a way to detect plant disease early on in the planting
season without investing too much time manually checking everything
themselves. 

They need software and hardware tools to do plant disease
detection at scale to be more cost-effective in the planting process.
In other words, there must be a classifier built that can process
an image of a plant and detect whether a plant is diseased or not and what
disease it has.

Access to reliable cellular and/or wifi data is often limited in rural areas,
so any kind of tool for this problem area must also be able to run offline on cheap
embedded systems built specifically for machine learning, e.g. NVIDIA's Jetson Nano: 

\url{https://developer.nvidia.com/embedded/jetson-nano-developer-kit}

\section{Dataset and Input}

The main dataset used for training and validation will be the PlantVillage
Dataset obtained from GitHub:

\url{https://github.com/spMohanty/PlantVillage-Dataset}

The dataset consists of 20,638 images each of which have the dimensions
256 by 256 pixels and are all colored, non-greyscale JPEGs. The images
are split up into several folders, labelled according to the plant
species (e.g. Pepper Bell) and whether they are healthy or not (e.g.
Pepper bell healthy, Pepper bell Bacterial spot).

\section{Solution Statement}

To address the need for a plant disease detecting tool, I will be
building a Python library with an API that will have two major functions:
image preprocessing necessary to prepare the data for classification,
and the classifier itself. This solution is based on the work done in some previous studies \cite{tripathi_recent_2016}\cite{sladojevic_deep_2016}.

The classifier will be built with using a Convolutional Neural Network (CNN)
with TensorFlow. This model will be based on the CNNs described in previous studies \cite{fuentes_robust_2017}\cite{toda_how_2019}.

The library will be used in a web app that will act as a UI
for the trained classifier using the Flask framework. It will allow
the user to upload photos to be preprocessed in the backend and 
then sent to the endpoint for classification.

A functioning prototype of the plant disease detector will be built 
using a Jetson Nano robot equipped with a camera. This will act as a proof-of-concept as to what
can be achieved using open source machine learning and cheap embedded systems.

\section{Benchmark Model}

The benchmark for the classifier will be the results obtained by previous studies by
Toda et al, Fuentes et al, on plant detection which all utilize a CNN as a classifier\cite{toda_how_2019} \cite{fuentes_robust_2017}.

\section{Evaluation Metrics}

The metrics used to evaluate the performance of the model will be
the accuracy of the model when it comes to the predicted classes vs
the actual ones, confusion matrix, and F-score.

\section{Project Design}

\subsection{Preprocessing Data}

The images will be preprocessed according to methodologies similar to those described in previous studies\cite{al_hiary_fast_2011}\cite{toda_how_2019}\cite{fuentes_robust_2017}: (1) color transformation of the RGB values in the image, (2) masking of the non-diseased parts of the leaf.

\subsection{Training Model}

The plant disease classifier will be built using TensorFlow, with
the model being a Convolutional Neural Network (CNN) trained on the
image dataset of 256x256 images in the PlantVillage dataset. This type of model was chosen
because CNNs have shown promising results in disease classification
in previous studies on this topic\cite{sladojevic_deep_2016}\cite{toda_how_2019}.

\subsection{Deployment}

The trained model will be imported into AWS SageMaker so that it can be deployed as an endpoint. The Flask web app will act as a front-end for that endpoint. It will have an upload screen to allow the user to submit a photo of a plant leaf and determine what disease it has, if any.

\bibliographystyle{IEEEtran}
\bibliography{citations}

\end{document}
