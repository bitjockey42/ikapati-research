%% LyX 2.3.3 created this file.  For more info, see http://www.lyx.org/.
%% Do not edit unless you really know what you are doing.
\documentclass[11pt]{article}
\usepackage[latin9]{inputenc}
\usepackage{url}
\usepackage{amsmath}

\makeatletter
%%%%%%%%%%%%%%%%%%%%%%%%%%%%%% User specified LaTeX commands.
%Gummi|065|=)
\usepackage{url}
\title{\textbf{Capstone Project}}
\author{Allyson Julian}
\date{January 27, 2020}

\makeatother

\begin{document}
\maketitle

\section{Definition}

\subsection{Project Overview}

Advances in agricultural technology over the past 30 years have made
it easier for farmers to manage their farms, particularly when the
farms are comprised of multiple fields greater than 1000 acres in
size. The use of GPS on aerial imagery, drones, etc. have been instrumental
in precision agriculture \cite{liakos_machine_2018}.

But one of the challenges of modern farming is the detection of plant
diseases. For large farms in particular, it is time-consuming for
a farmer to manually check each growing plant for disease. It can
potentially be more cost-effective to diagnose plant diseases with
automated tools \cite{fujita_basic_2016}.

For this project, I built a Python library that implements a multi-class classifier that functions as a plant disease detector.

The library includes functionality to do many of the tasks required in machine learning research, among them image preprocessing necessary to prepare the data for classification, scripts to train classifier itself, and utility functions to help evaluate performance.

\subsection{Problem Statement}

My main objectives for this project were to build a library that implements plant disease classification and can act as a framework with which to do further research in this subject area. It can be used to complete an entire machine learning pipeline from training to deployment.

To build this library, I needed to:

\begin{enumerate}
	\item Retrieve the PlantVillage Dataset (\url{https://github.com/spMohanty/PlantVillage-Dataset}).
	\item Prepare the raw color images from the PlantVillage Dataset for consumption by the model.
	\item Train the model to classify the different diseases for a given species of plant.
	\item Save the model as a TensorFlow Lite object for use in an embedded system.
\end{enumerate}

\subsection{Metrics}

Accuracy and loss were the main metrics used in this project.

\section{Analysis}

\subsection{Data Exploration}

The main datasets used were obtained from the PlantVillage Dataset: \url{https://github.com/spMohanty/PlantVillage-Dataset}

The dataset consists of 20,638 images each of which have the dimensions
256 by 256 pixels and are all colored, non-greyscale JPEGs. The images
are split up into several folders, labelled according to the plant
species (e.g. Pepper Bell) and whether they are healthy or not (e.g.
Pepper bell healthy, Pepper bell Bacterial spot).

In this project, the colored 

\subsection{Exploratory Visualization}

\subsection{Algorithms and Techniques}

\subsection{Benchmark}

The benchmark for the classifier will be the results obtained by previous studies by
Toda et al, Fuentes et al, on plant detection which all utilize a CNN as a classifier\cite{toda_how_2019} \cite{fuentes_robust_2017}.

\section{Methodology}

\subsection{Data Preprocessing}

Preprocessing the data was completed using these steps:

\begin{enumerate}
	\item Read in image filenames as list and shuffle items.
	\item Split image dataset into training (80\%), validation (10\%), and test (10\%) datasets.
	\item Write each dataset into TensorFlow records that contain the raw image and labels as the features.
	\item Create parser function to read each TensorFlow record batch by batch.
	\item Normalize image pixel values by dividing by 255 (representing the range of values for an RGB image).
\end{enumerate}

\subsection{Implementation}

\subsection{Refinement}

\section{Results}


\subsection{Justification}

\section{Conclusion}

\subsection{Visualization}

\subsection{Reflection}

\subsection{Improvement}

\bibliographystyle{IEEEtran}
\bibliography{citations}

\end{document}

\subsection{Model Evaluation and Validation}