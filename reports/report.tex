%% LyX 2.3.3 created this file.  For more info, see http://www.lyx.org/.
%% Do not edit unless you really know what you are doing.
\documentclass[11pt]{article}
\usepackage[latin9]{inputenc}
\usepackage{url}
\usepackage{amsmath}

\makeatletter
%%%%%%%%%%%%%%%%%%%%%%%%%%%%%% User specified LaTeX commands.
%Gummi|065|=)
\usepackage{url}
\title{\textbf{Capstone Project}}
\author{Allyson Julian}
\date{January 27, 2020}

\makeatother

\begin{document}
\maketitle

\section{Definition}

\subsection{Project Overview}

One of the greatest challenges in modern farming is the detection of plant diseases. Farmers need a way to detect plant disease early on in the planting season without investing too much time manually checking everything themselves.

For this project, I built a Python library that implements a multi-class classifier that functions as a plant disease detector.

The library includes functionality to do many of the tasks required in machine learning research, among them image preprocessing necessary to prepare the data for classification, scripts to train classifier itself, and utility functions to help evaluate performance.

\subsection{Problem Statement}

My main objectives for this project were to build a library that implements plant disease classification and can act as a framework with which to do further research in this subject area. It can be used to complete an entire machine learning pipeline from training to deployment.

To build this library, I needed to:

\begin{enumerate}
	\item Retrieve the PlantVillage Dataset (\url{https://github.com/spMohanty/PlantVillage-Dataset}).
	\item Prepare the raw color images from the PlantVillage Dataset for consumption by the model.
	\item Train the model to classify the different diseases for a given species of plant.
	\item Save the model as a TensorFlow Lite object for use in an embedded system.
\end{enumerate}

\subsection{Metrics}

Accuracy and loss were the main metrics used in this project.

\section{Analysis}

\subsection{Data Exploration}

\subsection{Exploratory Visualization}

\subsection{Algorithms and Techniques}

\subsection{Benchmark}

\section{Methodology}

\subsection{Data Preprocessing}

To preprocess the images, 

\subsection{Implementation}

\subsection{Refinement}

\section{Results}

\subsection{Model Evaluation and Validation}

\subsection{Justification}

\section{Conclusion}

\subsection{Visualization}

\subsection{Reflection}

\subsection{Improvement}

\bibliographystyle{IEEEtran}
\bibliography{citations}

\end{document}
